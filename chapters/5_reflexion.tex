\chapter{Reflexion}
Diese Arbeit wurde mit den Fragestellungen \enquote{Digitalisierung und Energieeffizienz, passen diese beiden Programme denn überhaupt zusammen?} und \enquote{Lässt sich die schnelllebige digitale Welt mit den extrem langlebigen Erneuerungszyklen beim Gebäudebestand koppeln um sowohl den individuellen Anforderungenan Komforterhöhung und Energie -Effizienzsteigerung bei der Gebäudenutzung gerecht zuwerden?} eingeleitet. Abschließend müssen diese Fragen mit einem \textbf{Ja} beantwortet werde, in der Theorie zumindest.

Es konnte dargelegt werden, dass es möglich sein kann, in die \gls{TGA} eines Gebäudes einzugreifen und dieses weitestgehend autonom mit Hilfe einer \gls{KI} zu steuern. Dabei war die ursprüngliche Zielsetzung neben der theoretischen Ausarbeitung und Analyse der Daten diese smarten Geräte tatsächlich praktisch miteinander kommunizieren zu lassen um mit einer \gls{KI} Optimierungen vorzunehmen. Mit der zunehmenden Betrachtung der Daten und des gesamten Szenarios wurde jedoch schnell klar, dass eine \enquote{blinde} Umsetzung einer \gls{KI} wenig zielführend sein würde. Die Breite des Themengebiets streckt sich, wie dargestellt, auf viele verschiedene Bereiche der Gebäudesteuerung und Analytik von Daten. In dem Zusammenhang kann fast schon von Big Data gesprochen werden.

Diese Arbeit bietet eine Grundlage für folgende Forschungen auf dem Gebiet des Einsatzes von \gls{KI} im Bereich der Gebäudetechnik.