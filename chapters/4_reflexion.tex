\chapter{Reflexion}
Trotz der nicht vollständig abgeschlossenen Projekte wurde die vierte Praxisphase bei der Siemens AG erfolgreich abgeschlossen. Das erworbene Wissen und die angefangenen Arbeiten dienen als Grundlage für weitere Entwicklungen im Bereich des MES Portals und weiteren Produkten bezüglich \gls{OP EX PH}. Eine Portierung weiterer Produkte von \gls{VB}6 in webbasierte Technologien ist absehbar. In diesem Bereich dienen Erfahrungen mit neuen Frameworks als Grundlage für den schnelleren Erfolg weiterer Entwicklungen.

\section{Informationsgewinn \& Lernerfolge}
Die Analyse des \gls{VB}6 Codes (siehe \Cref{subsec_backend}) ermöglichte Einblicke in das Verstehen von fremden Code in einer fremden Programmiersprache, ohne mit dieser Sprache selbst programmieren zu müssen. Aufgrund des konstanten Wandels und der stetigen Weiterentwicklung von Technologien, wird diese Fähigkeit auch in Zukunft Anwendung finden.\\
Aus programmiertechnischer Sicht konnte die neue Web Framework Blazor kennengelernt werden, das im Jahr 2020 sehr großen Anklang gefunden hat und eine neue alternative zu JavaScript basierten Frameworks wie Angular oder React ist. Des Weiteren wurde mit C\myHashtag\ \gls{full-stack} entwickelt und der Umgang mit OracleSQL wurde vertieft.\\
Für das kollaborative Arbeiten wurde das Versionskontrollsystem GIT verwendet, was das am weitesten verbreitete Versionskontrollsystem ist (vgl. \cite{datanyze.2020a}, unter der Beachtung, dass z. B. Github auf GIT basiert).

\section{Überschneidung mit Vorlesungsinhalten}
Durch die detaillierten Einblicke in Datenbanksysteme und in die \gls{SQL} durch die Vorlesung \enquote{Datenbanken} des dritten und vierten Semesters, konnte die Arbeit mit OracleSQL stark erleichtert werden.\\
Das Arbeiten mit neuen Technologien und generelle Arbeit in der Softwareentwicklung wurde in der Vorlesung \enquote{Software Engineering I} behandelt. Für die Vorlesung \enquote{Software Engineering II} konnten hier sinnvolle Kenntnisse angeeignet werden.\\
Obwohl es sich um ein Webentwicklungsprojekt handelt, konnte das Wissen aus den Vorlesungen \enquote{Webengineering I + II} hier nicht angewandt werden. Der Grund dafür ist, dass Blazor nicht wie andere Web Technologien funktioniert und in diesem Sinn nicht in den Vorlesungen behandelt wurde. Durch Radzen wurde die Arbeit mit HTML, CSS und JavaScript abgenommen.