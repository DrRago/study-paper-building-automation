\chapter{Energieeffizienz}
Besonders im Wohnungsbau in Ballungsräumen ist es entscheidend, die bestehenden Energieeffizienzpotentiale ohne große zusätzliche Investitionen zu heben, um zusätzliche Belastungen der Eigentümer und der Mieter durch Mietumlagen zu verhindern.  Die beschriebene Anwendung smarter Geräte bei der Gebäudetechnik aber auch des täglichen Lebens ist vielversprechend, da Effizienzpotentiale durch Betriebsoptimierung und intelligente Clusterkopplung (siehe Tabelle der smarten Geräte \Cref{fig-smart-devices}) erreicht werden können, ohne hohe Investitionen zu erfordern. Die beschriebenen Lösungsansätze können auf ein breites Gebäudeportfolio übertragen und damit deren flächige Anwendbarkeit und Adaption auf den deutschen Gebäudebestand gezeigt werden. Im Bereich der Datengrundlagen ist eine weitgehend automatisierte Datenanalyse und eine umfassende Digitalisierung des Gebäudebetriebs gerade in den wirtschaftlich und energetisch vielversprechenden Bereichen des optimierten dynamischen Betriebs sowie der intelligenten Clusterkopplung anzustreben. Die smarte Gebäudetechnologie erhöht nicht nur den Wohnwert, sondern führt aufgrund eines optimierten Betriebes zu einer direkte Kosteneinsparung für den Bewohner und Eigentümer. Aus diesem Grunde ist mit einer hohen Marktakzeptanz zu rechnen, was auf lange Sicht zu einer immer weiter voranschreitenden digitalen Durchdringung des Gebäudebestandes führen wird. Die Hersteller der smarten Geräte sind darüber hinaus bestrebt, hochwertige und einfach zu bedienende Technologien zu entwickeln, die plug-and-play-fähig sind und sich eigenständig in das Gebäudeleitsystem integrieren. Die smarten Geräte können sich damit selbst optimieren und die jeweiligen individuellen Betriebszustände nach minimalem Energieverbrauch aber auch nach maximalem Komfort anfahren. In beiden Fällen wird ein \enquote{Nutzerwunsch} ausgeführt, der, wenn eine \gls{KI} ihn einstellt, effizienter realisiert wird, als es durch einen Nicht-Experten möglich ist. Selbst wenn wie im Falle der dezentralen Heizungspumpe (WILO SE, Dortmund) je Heizkörper nur ein geringer Absolutbetrag an Energie eingespart wird, summiert sich das bundesdeutsche Gesamtpotential aufgrund der Vielzahl der Gebäude zu einem energiepolitisch relevanten Gesamtbetrag. Doch nicht nur Energie lässt sich mit dem Smart Home sparen: Die Gebäudedigitalisierung ermöglicht es, Veränderungen im Betrieb zu erkennen um so beispielsweise die Ausfallwahrscheinlichkeit technischer Geräte zu berechnen (Predictive Maintenance). So kann die Erhöhung des Stromverbrauchs einer Pumpe einer Hebeanlage Hinweise auf einen baldigen Ausfall, auf Rohrleitungsfouling oder auf einen hydraulischen Kurzschluss im Wasserversorgungsystem sein.  Manch feuchter Keller wird durch Leckagen in den Grundleitungen verursacht: Eine automatische Bilanzierung der Abwasser- und Frischwassermengen könnte dieses unter Erhöhung der Ressourceneffizienz aufdecken.  Eine der größten Herausforderungen wird angesichts der demographischen Struktur und der sich rasend schnell entwickelnden digitalen Technologien die Mensch-Maschine-Schnittstelle sein. Diese muss sicherstellen, dass der Anwender immer über den Betrieb seines Gebäudes informiert ist, dass er die digital entdeckten Optimierungspotentiale auch umsetzt und dass die ermittelten Gebäudeinformationen automatisiert zur Verfügung gestellt werden. Zur Beurteilung des Zustandes einzelner Komponenten, bestimmter Gebäudecluster oder des Gesamtzustands des Gebäudes hat sich ein Ampelsystem bewehrt:

\begin{itemize}
    \itemsep0em
    \item[] \textbf{Rot:} dringender Handlungsbedarf, im Vergleich zur Historie wird mehr Energie verbraucht,
    \item[] \textbf{gelb}: Historische Werte decken sich mit Ist-Werten $\rightarrow$ kein Handlungsbedarf und
    \item[] \textbf{grün:} der Ist-Zustand arbeitet energieeffizienter. 
\end{itemize}

Sind alle Optimierungspotentiale ausgeschöpft, dann gibt es keine Verbesserungen mehr, das Gesamtsystem pendelt sich folglich auf den \enquote{gelben} Zustand ein. Nun befindet sich das Gebäude was den Energieverbrauch betrifft zumindest in einem lokalen Minimum.
