%%%%%%%%%%%%%%%%%%%%%%%%%%%%%%%%%%%%%%%%%
% fphw Assignment
% LaTeX Template
% Version 1.0 (27/04/2019)
%
% This template originates from:
% https://www.LaTeXTemplates.com
%
% Authors:
% Class by Felipe Portales-Oliva (f.portales.oliva@gmail.com) with template 
% content and modifications by Vel (vel@LaTeXTemplates.com)
%
% Template (this file) License:
% CC BY-NC-SA 3.0 (http://creativecommons.org/licenses/by-nc-sa/3.0/)
%
%%%%%%%%%%%%%%%%%%%%%%%%%%%%%%%%%%%%%%%%%

%----------------------------------------------------------------------------------------
%	PACKAGES AND OTHER DOCUMENT CONFIGURATIONS
%----------------------------------------------------------------------------------------

\documentclass[
	ngerman,
	12pt, % Default font size, values between 10pt-12pt are allowed
	%letterpaper, % Uncomment for US letter paper size
	%spanish, % Uncomment for Spanish
]{pm}

% Template-specific packages
\usepackage[utf8]{inputenc} % Required for inputting international characters
\usepackage[T1]{fontenc} % Output font encoding for international characters
\usepackage{mathpazo} % Use the Palatino font

\usepackage{graphicx} % Required for including images

\usepackage{booktabs} % Required for better horizontal rules in tables

\usepackage{listings} % Required for insertion of code

\usepackage{enumerate} % To modify the enumerate environment

\usepackage[ngerman]{babel}
\usepackage{csquotes}
\usepackage{tabularx}

%----------------------------------------------------------------------------------------
%	ASSIGNMENT INFORMATION
%----------------------------------------------------------------------------------------

\title{Künstliche Intelligenz im Bereich der Gebäudetechnik}

\newcommand{\task}{Qualitätssicherung} % Assignment title

\author{Leonhard Gahr} % Student name

\date{23. März 2021} % Due date

\institute{\includegraphics[width=5cm]{../img/sie-logo.png}\hfill\includegraphics[width=4cm]{../img/dhbw-logo}} % Institute or school name

\class{TINF18B4} % Course or class name

\professor{Michael Vetter} % Professor or teacher in charge of the assignment

%----------------------------------------------------------------------------------------

\begin{document}

\maketitle % Output the assignment title, created automatically using the information in the custom commands above

%----------------------------------------------------------------------------------------
%	ASSIGNMENT CONTENT
%----------------------------------------------------------------------------------------

\section*{\large Maßnahmen}
Wie in der Projektplanung beschrieben besteht die Qualitätssicherung lediglich aus einer korrektur der Form und Sprache der Arbeit, da der Rahmen dieses Projekts keine weiteren, tiefgründigeren Maßnahmen erlaubt.\\
Hierbei werden die Kapitel der Arbeit abschnittsweise von unabhängigen dritten korrekturgelesen und Formulierungsverbesserungen sowie Rechtschreibfehler aufgedeckt und korrigiert. Die Korrekturempfehlungen werden telefonisch abgesprochen und korrigiert.\\
Da die Arbeit auf der inhaltlichen Ausarbeitung basiert, die sehr komplexe Forschungsfelder abdecken, kann die Qualität der Inhalte nicht von Experten auf dem Gebiet bewertet werden. Dies ist im Endeffekt die Grundlage der Bewertung des Projekts durch den Gutachter der Studienakademie.

\end{document}