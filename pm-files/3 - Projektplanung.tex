%%%%%%%%%%%%%%%%%%%%%%%%%%%%%%%%%%%%%%%%%
% fphw Assignment
% LaTeX Template
% Version 1.0 (27/04/2019)
%
% This template originates from:
% https://www.LaTeXTemplates.com
%
% Authors:
% Class by Felipe Portales-Oliva (f.portales.oliva@gmail.com) with template 
% content and modifications by Vel (vel@LaTeXTemplates.com)
%
% Template (this file) License:
% CC BY-NC-SA 3.0 (http://creativecommons.org/licenses/by-nc-sa/3.0/)
%
%%%%%%%%%%%%%%%%%%%%%%%%%%%%%%%%%%%%%%%%%

%----------------------------------------------------------------------------------------
%	PACKAGES AND OTHER DOCUMENT CONFIGURATIONS
%----------------------------------------------------------------------------------------

\documentclass[
	ngerman,
	12pt, % Default font size, values between 10pt-12pt are allowed
	%letterpaper, % Uncomment for US letter paper size
	%spanish, % Uncomment for Spanish
]{pm}

% Template-specific packages
\usepackage[utf8]{inputenc} % Required for inputting international characters
\usepackage[T1]{fontenc} % Output font encoding for international characters
\usepackage{mathpazo} % Use the Palatino font

\usepackage{graphicx} % Required for including images

\usepackage{booktabs} % Required for better horizontal rules in tables

\usepackage{listings} % Required for insertion of code

\usepackage{enumerate} % To modify the enumerate environment

\usepackage[ngerman]{babel}
\usepackage{csquotes}

%----------------------------------------------------------------------------------------
%	ASSIGNMENT INFORMATION
%----------------------------------------------------------------------------------------

\title{Künstliche Intelligenz im Bereich der Gebäudetechnik}

\newcommand{\task}{Projektplanung} % Assignment title

\author{Leonhard Gahr} % Student name

\date{13. März 2021} % Due date

\institute{\includegraphics[width=5cm]{../img/sie-logo.png}\hfill\includegraphics[width=4cm]{../img/dhbw-logo}} % Institute or school name

\class{TINF18B4} % Course or class name

\professor{Michael Vetter} % Professor or teacher in charge of the assignment

%----------------------------------------------------------------------------------------

\begin{document}

\maketitle % Output the assignment title, created automatically using the information in the custom commands above

%----------------------------------------------------------------------------------------
%	ASSIGNMENT CONTENT
%----------------------------------------------------------------------------------------

\section*{\large Terminplanung}
\begin{enumerate}
	\item [12.05.2021] \quad Präsentation der Inhalte der Arbeit vor dem Kurs
	\item [17.05.2021] \quad Abgabe der Arbeit in textueller Form online sowie gedruckt und gebunden
\end{enumerate}


\section*{\large Meilensteine}
\begin{enumerate}
	\item [01.04.2021] \quad Grundlagen geklärt - Was ist eine künstliche Intelligenz
	\item [10.04.2021] \quad Modellierungstheorie einer künstlichen Intelligenz geklärt
	\item [17.04.2021] \quad Methode zur Datensammlung unterschiedlichster Quellen geklärt
	\item [24.04.2021] \quad Modellieren eines neuronalen Netzes für den Anwendungsfall
	\item [28.04.2021] \quad Faktor Energieeffizienz ermittelt und ausgearbeitet
	\item [01.05.2021] \quad Einleitung geschrieben
	\item [07.05.2021] \quad Reflexion und Zukunftsassichten für die Arbeit ermittelt
	\item [12.05.2021] \quad Präsentation erstellt
	\item [17.05.2021] \quad Übergabe der Arbeit an den Betreuer
\end{enumerate}

\section*{\large Qualitätssicherung}
Die Qualitätssicherung in dieser theoretischen Ausarbeitung besteht lediglich aus der Überprüfung der Form und Sprache der textuellen Ausarbeitung, die im Rahmen der Durchführunng genauer erläutert wird.

\end{document}
