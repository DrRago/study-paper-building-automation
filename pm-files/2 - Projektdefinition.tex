%%%%%%%%%%%%%%%%%%%%%%%%%%%%%%%%%%%%%%%%%
% fphw Assignment
% LaTeX Template
% Version 1.0 (27/04/2019)
%
% This template originates from:
% https://www.LaTeXTemplates.com
%
% Authors:
% Class by Felipe Portales-Oliva (f.portales.oliva@gmail.com) with template 
% content and modifications by Vel (vel@LaTeXTemplates.com)
%
% Template (this file) License:
% CC BY-NC-SA 3.0 (http://creativecommons.org/licenses/by-nc-sa/3.0/)
%
%%%%%%%%%%%%%%%%%%%%%%%%%%%%%%%%%%%%%%%%%

%----------------------------------------------------------------------------------------
%	PACKAGES AND OTHER DOCUMENT CONFIGURATIONS
%----------------------------------------------------------------------------------------

\documentclass[
	ngerman,
	12pt, % Default font size, values between 10pt-12pt are allowed
	%letterpaper, % Uncomment for US letter paper size
	%spanish, % Uncomment for Spanish
]{pm}

% Template-specific packages
\usepackage[utf8]{inputenc} % Required for inputting international characters
\usepackage[T1]{fontenc} % Output font encoding for international characters
\usepackage{mathpazo} % Use the Palatino font

\usepackage{graphicx} % Required for including images

\usepackage{booktabs} % Required for better horizontal rules in tables

\usepackage{listings} % Required for insertion of code

\usepackage{enumerate} % To modify the enumerate environment

\usepackage[ngerman]{babel}
\usepackage{csquotes}

%----------------------------------------------------------------------------------------
%	ASSIGNMENT INFORMATION
%----------------------------------------------------------------------------------------

\title{Künstliche Intelligenz im Bereich der Gebäudetechnik}

\newcommand{\task}{Projektdefinition} % Assignment title

\author{Leonhard Gahr} % Student name

\date{08. März 2021} % Due date

\institute{\includegraphics[width=5cm]{../img/sie-logo.png}\hfill\includegraphics[width=4cm]{../img/dhbw-logo}} % Institute or school name

\class{TINF18B4} % Course or class name

\professor{Michael Vetter} % Professor or teacher in charge of the assignment

%----------------------------------------------------------------------------------------

\begin{document}

\maketitle % Output the assignment title, created automatically using the information in the custom commands above

%----------------------------------------------------------------------------------------
%	ASSIGNMENT CONTENT
%----------------------------------------------------------------------------------------

\section*{\large Definition}
Wie der Aufgabenstellung zu entnehmen ist, ist das Ziel des Proejkts die theoretische Ausarbeitung eines Programms, welches mithilfe einer künstlichen Intelligenz Geräte im Smart Home steuern kann, ohne explizit menschliche Anfragen durchführen zu müssen.\\
Hierfür gilt als grundlegender Forschungsansatz die Ausarbeitung der Funktionsweise von künstlichen Intelligenzen, insbesondere neuronaler Netzen, sowie die möglichkeiten für die Evolution (verbesserung) dieser.\\
Im Rahmen der Arbeit sollen Aspekte wie Datenintegration und resultierende Vorteile (Energieeffizienz, Komfortgewinnung, etc.) ausarbeitet und erläutert werden.\\ 
Diese Ausarbeitung soll die Grundlage für die Entwicklung eines Systems sein, das mit reellen Daten tatsächlich existierende Geräte ansteuert.

\section*{\large Organisation}
\begin{enumerate}
	\item [--] Abschluss: 17.05.2021
	\item [--] Budget: 0,00€
	\item [--] Team: Alleinarbeit
	\item [--] Technologie zur Dokumentenverfassung: \LaTeX
\end{enumerate}



\end{document}
