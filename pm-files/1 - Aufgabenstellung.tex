%%%%%%%%%%%%%%%%%%%%%%%%%%%%%%%%%%%%%%%%%
% fphw Assignment
% LaTeX Template
% Version 1.0 (27/04/2019)
%
% This template originates from:
% https://www.LaTeXTemplates.com
%
% Authors:
% Class by Felipe Portales-Oliva (f.portales.oliva@gmail.com) with template 
% content and modifications by Vel (vel@LaTeXTemplates.com)
%
% Template (this file) License:
% CC BY-NC-SA 3.0 (http://creativecommons.org/licenses/by-nc-sa/3.0/)
%
%%%%%%%%%%%%%%%%%%%%%%%%%%%%%%%%%%%%%%%%%

%----------------------------------------------------------------------------------------
%	PACKAGES AND OTHER DOCUMENT CONFIGURATIONS
%----------------------------------------------------------------------------------------

\documentclass[
	ngerman,
	12pt, % Default font size, values between 10pt-12pt are allowed
	%letterpaper, % Uncomment for US letter paper size
	%spanish, % Uncomment for Spanish
]{pm}

% Template-specific packages
\usepackage[utf8]{inputenc} % Required for inputting international characters
\usepackage[T1]{fontenc} % Output font encoding for international characters
\usepackage{mathpazo} % Use the Palatino font

\usepackage{graphicx} % Required for including images

\usepackage{booktabs} % Required for better horizontal rules in tables

\usepackage{listings} % Required for insertion of code

\usepackage{enumerate} % To modify the enumerate environment

\usepackage[ngerman]{babel}
\usepackage{csquotes}

%----------------------------------------------------------------------------------------
%	ASSIGNMENT INFORMATION
%----------------------------------------------------------------------------------------

\title{Künstliche Intelligenz im Bereich der Gebäudetechnik}

\newcommand{\task}{Aufgabenstellung} % Assignment title

\author{Leonhard Gahr} % Student name

\date{\today} % Due date

\institute{\includegraphics[width=5cm]{../img/sie-logo.png}\hfill\includegraphics[width=4cm]{../img/dhbw-logo}} % Institute or school name

\class{TINF18B4} % Course or class name

\professor{Michael Vetter} % Professor or teacher in charge of the assignment

%----------------------------------------------------------------------------------------

\begin{document}

\maketitle % Output the assignment title, created automatically using the information in the custom commands above

%----------------------------------------------------------------------------------------
%	ASSIGNMENT CONTENT
%----------------------------------------------------------------------------------------

\section*{\large Ziel}
Die Studienarbeit zum Thema \enquote{Künstliche Intelligenz im Bereich der Gebäudetechnik} behandelt die Theorie hinter der Automatisierung von Prozessen im SmartHome. Dabei soll erforscht werden, inwiefer ein Programm dazu in der Lage ist, von selbst Routinen zu erlernen.\\
Dabei werden, neben der reinen \enquote{Kompfortgewinnung}, auch die Aspekte der Energieeffiznez und Umweltfreundlichkeit berücksichtigt.\\\\
Die Arbeit wird in kompletter Eigenarbeit, ohne zutun des Betreuers, erstellt.

\section*{\large Nicht-Ziel}
Die Studienarbeit behandelt ausdrücklich die Theorie hinter einer möglichen Entwicklung eines Programmes. Die konkrete Umsetzung gehört nicht zum Rahmen der Arbeit.

\section*{\large Zusätzliches}
Wie in der Themenmitteilung (siehe Anhang) angegeben ist, werden insbesondere die Werke \enquote{Grundkurs Künstliche Intelligenz} von Wolfgang Ertel sowie \enquote{Leben 3.0: Mensch sein im Zeitalter Künstlicher Intelligenz} von Max Tegmark als Grundlage der Recherche verwendet.


\end{document}
